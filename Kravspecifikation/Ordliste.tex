\subsection{Ordliste (Magnus)}
 %Dette afsnit skal indeholde en opremsning af de væsentlige begreber fra problem domænet. Hvert begreb gives en kort definition på nogle få linier.

De væsentligste begreber fra problemdomænet er defineret som følgende.

\begin{itemize}
\item \textbf{Projektplanlægger:} Et software system til timeregistrering og projektstyring. Systemet bruges internt. Brugerne er medarbejdere i en virksomhed kaldet Softwarehuset A/S.

\item \textbf{Projekt:} Repræsenterer virksomhedens opgaver. Projekter er inddelt i cirka 30 til 100 aktiviteter.

\item \textbf{Aktivitet:} Repræsenterer blokke af timer, en medarbejder bruger. Der er to typer aktiviteter: projekteraktiviteter og statusaktiviter. Grafisk ses de som farvede blokke i projektplanlæggerens kalender.

\item \textbf{Projektaktivitet:} De mindste dele af et projekt. Oprettes af projektledere. En projektaktivitet er tilknyttet én eller flere medarbejdere. De tilknyttede medarbejde kan søge assistance hos andre medarbejdere. Medarbejdere tilknyttes en projektaktivitet af en projektleder.

\item \textbf{Statusaktivitet:} En særlig type aktivitet, der repræsenterer når en medarbejder er syg eller på ferie. Medarbejder kan selv tilknytte sig statusaktiviteter, dvs. man melder sig selv syg.

\item \textbf{Medarbejder:} Én af cirka 50 ansatte i i Softwarehuset A/S. Medarbejdere tilføjes til projekter af projektledere. Medarbejdere kan ikke selv oprette aktiveter. Medarbejdere kan søge assistance til aktiviteter, de arbejder på. Medarbejdere arbejder typisk på mindre end 10 aktiviteter ad gangen.

\item \textbf{Projektleder:} Hvert projekt har en projektleder. En projektleder er en medarbejder, der kan tilføje, ændre og slette aktiviteter for det projekt, han er projektleder for. Når en medarbejder opretter et projekt, bliver han automatisk markeret som projektleder. Den nuværende projektleder kan give denne status videre til en anden medarbejder, der er tilknyttet projektet. (se evt. sekvensdiagrammerne).

\end{itemize}


